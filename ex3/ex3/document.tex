\documentclass[11pt, a4paper]{article}

% font time new roman
\usepackage{fontspec}
\setmainfont{Times New Roman}

% go tieng viet
%\usepackage[utf8]{vietnam}

% thu vien toan hoc
\usepackage{amsmath}
\usepackage{amssymb}

% thu vien mau
\usepackage{color, xcolor}

% thu vien chi dan
\usepackage{hyperref}

% list package
\usepackage{enumerate}

% underline text
\usepackage{ulem}

% code environment
\usepackage{listings}

%%%%%%%%%%%%%%%%%%%%%%%%%%%%%%%%%%%%%%%%%%%%%%%%%%%%%%%%
% config code environment
\definecolor{codegreen}{rgb}{0,0.6,0}
\definecolor{codegray}{rgb}{0.5,0.5,0.5}
\definecolor{codepurple}{rgb}{0.58,0,0.82}
\definecolor{backcolour}{rgb}{0.95,0.95,0.92}

\lstdefinestyle{base_style}{
%	backgroundcolor=\color{backcolour},		%background color   
	commentstyle=\color{codegreen},			%comment color
	keywordstyle=\color{magenta},			%keyword color
	numberstyle=\tiny\color{codegray},		%number style(mau, font)
	stringstyle=\color{codepurple},			%string color
	basicstyle=\ttfamily\footnotesize,		%family, fontsize
%	breakatwhitespace=false,         		%bo qua khoang trang
%	breaklines=true,                 		%cho phep break line
	captionpos=b,                    		%cap
%	keepspaces=true,                 
	numbers=left,                    
%	numbersep=10pt,                  
%	showspaces=false,                
	showstringspaces=false,
	showtabs=false,                  
	tabsize=3
}

%\lstset{style=python_style}
%\lstlistoflistings


\begin{document}
	\begin{center}
		{\color{red} \bfseries \Large
		SoDiTEC\\}
		Homework Latex 3
	\end{center}
	\section{Một số lưu ý}
	\section{Biểu diễn trên dòng chứa các ký tự khác}
	Cho \(f: X \rightarrow Y\) là một ánh xạ
	\begin{enumerate}[1.]
		\item Ánh xạ f được gọi là đơn ánh nếu
		\begin{itemize}
			\item Với mọi \(x_1 \neq x_2\) thì \(f(x_1) \neq f(x_2)\) hoặc
			\item Nếu \(f(x_1) = f(x_2) \) thì \( x_1 = x_2\)
		\end{itemize}
		\item Toàn ánh\\
		Ánh xạ \(f\) được gọi là toàn ánh nếu \(f(X) = Y\) hay với mỗi \(y \in Y\), tồn tại \(x\in X\) sao cho \(f(x)=y\). Nói cách khác, phương trình \(f(x)=y\) có nghiệm với mọi \(y\in Y\)
		\item Song ánh\\
		Ánh xạ được gọi là song ánh nếu nó vừa là đơn ánh, vừa là toàn ánh. Nói cách khác, phương trình \(f(x) = y\) có nghiệm với mọi \(y\in Y\)
	\end{enumerate}
	\section{Biểu diễn ở một dòng riêng}
	\textbf{Ví dụ:} Cho hai ánh xạ
	\begin{align*}
		f:\mathbb{R}\backslash 0 &\rightarrow \mathbb{R}\\
		x &\rightarrow \frac{1}{x}
	\end{align*}
	\begin{enumerate}[a)]
		\item Ánh xạ nào là đơn ánh, toàn ánh. Tìm \(g(\mathbb{R})\)
		\item xác đinh ánh xạ \(h=g \circ f\)
	\end{enumerate}
	\textit{Lời giải:}
	\begin{enumerate}
		\item \(f\) là đơn ánh, không phải là toàn ánh, \(g\) không là đơn ánh, cũng không phải là toàn ánh
		\item \(g(\mathbb{R})=\left[-1,1\right]\)
	\end{enumerate}
	\begin{align}
		f(x) &= 2x^2+3x+6\\
		f(x_1) + f(x_2) &= 12x_1^2+8x_2^3+2x_1+7x_2+69\\
		f(x) + \frac{x^2}{x+3} &= \frac{x^3\pi}{x^3+5x^2+5x}\\
		f(x)&=\frac{x^3}{x^2+3x+6}\\
		f(A\cup B) &= f(A)\cup f(B), A, B \in X\\
		f(A\cap B)^{-1} &= f(A)^{-1}\cap f(B)^{-1}, A, B \in Y
	\end{align}
\end{document}
