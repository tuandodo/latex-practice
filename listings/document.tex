\documentclass[11pt, a4paper]{article}

% font time new roman
\usepackage{fontspec}
\setmainfont{Times New Roman}

% go tieng viet
%\usepackage[utf8]{vietnam}

% thu vien toan hoc
\usepackage{amsmath}
\usepackage{amssymb}

% thu vien mau
\usepackage{color, xcolor}

% thu vien chi dan
\usepackage{hyperref}

% list package
\usepackage{enumerate}

% underline text
\usepackage{ulem}

% code environment
\usepackage{listings}

%%%%%%%%%%%%%%%%%%%%%%%%%%%%%%%%%%%%%%%%%%%%%%%%%%%%%%%%
% config code environment
\definecolor{codegreen}{rgb}{0,0.6,0}
\definecolor{codegray}{rgb}{0.5,0.5,0.5}
\definecolor{codepurple}{rgb}{0.58,0,0.82}
\definecolor{backcolour}{rgb}{0.95,0.95,0.92}

\lstdefinestyle{base_style}{
%	backgroundcolor=\color{backcolour},		%background color   
	commentstyle=\color{codegreen},			%comment color
	keywordstyle=\color{magenta},			%keyword color
	numberstyle=\tiny\color{codegray},		%number style(mau, font)
	stringstyle=\color{codepurple},			%string color
	basicstyle=\ttfamily\footnotesize,		%family, fontsize
%	breakatwhitespace=false,         		%bo qua khoang trang
%	breaklines=true,                 		%cho phep break line
	captionpos=b,                    		%cap
%	keepspaces=true,                 
	numbers=left,                    
%	numbersep=10pt,                  
%	showspaces=false,                
	showstringspaces=false,
	showtabs=false,                  
	tabsize=3
}

%\lstset{style=python_style}
%\lstlistoflistings


\begin{document}
	\section{Sử dụng thư viện listings}
	Truy cập vào link sau để học thêm \href{https://www.overleaf.com/learn/latex/Code_listing#Using_listings_to_highlight_code}{"Link"}\\
	Ví dụ:
	\begin{lstlisting}[style=base_style, language=Python, caption={Python example}]
import numpy as np

def incmatrix(genl1,genl2):
	m = len(genl1)
	n = len(genl2)
	M = None #to become the incidence matrix
	VT = np.zeros((n*m,1), int)  #dummy variable

	#compute the bitwise xor matrix
	M1 = bitxormatrix(genl1)
	M2 = np.triu(bitxormatrix(genl2),1) 

	for i in range(m-1):
	for j in range(i+1, m):
	[r,c] = np.where(M2 == M1[i,j])
	for k in range(len(r)):
	VT[(i)*n + r[k]] = 1;
	VT[(i)*n + c[k]] = 1;
	VT[(j)*n + r[k]] = 1;
	VT[(j)*n + c[k]] = 1;

	if M is None:
	M = np.copy(VT)
	else:
	M = np.concatenate((M, VT), 1)
	
	VT = np.zeros((n*m,1), int)

	return M
	\end{lstlisting}
	\begin{lstlisting}[style=base_style, language=C, caption=C example]
#include <iostream>
int main(){
	/*comment here*/
	std::cout << "Hello World" << std::endl;
	return 0;
}
	\end{lstlisting}
	\section{Thư viện toán học}
	\subsection{Ví dụ}
	\begin{enumerate}
		\item 
	\end{enumerate}
\end{document}
